%
\documentclass{beamer}
%%\documentclass[handout]{beamer}

\definecolor{headcolor}{rgb}{0,0.3,0.6}
%\renewcommand{\alert}[1]{\textcolor{headcolor}{#1}}
\renewcommand{\alert}[1]{\textcolor{blue}{#1}}
\newcommand{\red}[1]{\textcolor{red}{#1}}
\mode<presentation>
{
  \usetheme{default}
  %% \setbeamercovered{transparent}
  \usefonttheme{professionalfonts}
  \usefonttheme{structurebold}
  \usecolortheme[rgb={0,0.3,0.6}]{structure}
  %%  \usecolortheme[rgb={0,0,0.7}]{structure}
}

\usepackage[english]{babel}
\usepackage[latin1]{inputenc}
\usepackage{times}
\usepackage[T1]{fontenc}
\usepackage{rotate}
\usepackage{color}
\usepackage{verbatim}
\usepackage{xmpmulti}
\usepackage{pgfarrows}
%\usepackage{pgf}

%\input{/home/tom/lib/tex/defs}
%%%%%%%%%%%%%%%%%%%%%%%%%%%%%%%%%%%%

% Delete this, if you do not want the table of contents to pop up at
% the beginning of each subsection:
%\AtBeginSection[]
%{
%  \begin{frame}<beamer>
%    \frametitle{Outline}
%    \tableofcontents[currentsection]
%  \end{frame}
%}
% \AtBeginSubsection[]
% {
%   \begin{frame}<beamer>
%     \frametitle{Outline}
%     \tableofcontents[currentsubsection]
%   \end{frame}
% }

% If you wish to uncover everything in a step-wise fashion, uncomment
% the following command: 
%\beamerdefaultoverlayspecification{<+->}

%%%%%%%%%%%%%%%%%%%%%%%%%%%%%%%%%%%%%%%%%%%%%%%%%%%%%%%%%%%%%%%%%%%%%%%%%%%%%%%

\def\width{12.5} % slide width
\def\height{8.5} % slide height
\newcommand{\showgrid}{%
  \pgfsetlinewidth{0.8pt} 
  \pgfgrid[step={\pgfpoint{1cm}{1cm}}]{\pgforigin}{\pgfxy(\width,\height)}{} 
  \pgfsetlinewidth{0.1pt} 
  \pgfgrid[stepx=0.1cm,stepy=0.1cm]{\pgforigin}{\pgfxy(\width,\height)}
}
\newcommand{\pgfslide}[1]{%
  \hspace*{-1cm}
  \begin{pgfpicture}{0cm}{0cm}{\width cm}{\height cm}
    #1
  \end{pgfpicture}  
}
\def\figpath{figs/}


%%%%%%%%%%%%%%%%%%%%%%%%%%%%%%%%%%%%
\newcommand{\A}{{\text{A}^{-}}}
\newcommand{\Ca}{{\text{Ca}^{2+}}}
\newcommand{\Cl}{{\text{Cl}^{-}}}
\newcommand{\di}{\text{d}}
\newcommand{\diff}[2]{\displaystyle\frac{\text{d}#1}{\text{d}#2}}
\newcommand{\e}{\text{e}}
\newcommand{\Epop}{\mathcal{E}}
\newcommand{\EW}[2][]{\text{E}_{#1}\left[#2\right]}
\newcommand{\Ex}{\text{E}}
\newcommand{\FT}{\mathfrak{F}}
\newcommand{\iFT}{\mathfrak{F}^{-1}}
\newcommand{\Fourier}[2]{\mathfrak{F}\left[#1\right]\left(#2\right)}
\newcommand{\iFourier}[2]{\mathfrak{F}^{-1}\left[#1\right]\left(#2\right)}
\newcommand{\In}{\text{I}}
\newcommand{\inp}{\text{inp}}
\newcommand{\Ipop}{\mathcal{I}}
\newcommand{\K}{{\text{K}^{+}}}
\newcommand{\Leak}{{\text{L}}}
\newcommand{\mm}{\text{mm}}
\newcommand{\mV}{\text{mV}}
\newcommand{\Na}{{\text{Na}^{+}}}
\newcommand{\one}{\mathbb{I}}
\newcommand{\rest}{\text{rest}}
\newcommand{\tauM}{\tau_\text{m}}
\newcommand{\tauR}{\tau_\text{ref}}
\newcommand{\vect}[2]{\begin{pmatrix}#1\\#2\end{pmatrix}}
%%%%%%%%%%%%%%%%%%%%%%%%%%%%%%%%%%%%

%%%%%%%%%%%%%%%%%%%%%%%%%%%%%%%%%%%%%%%%%%%%%%%%%%%%%%%%%%%%%%%%%%%%%%%%%%%%%%%
\begin{document}
%%
%% title page
\begin{frame}
  \pdfbookmark[2]{Title}{TitlePage}
  \pgfslide{
    %% 
    \pgfputat{\pgfxy(6.25,8)}{\pgfbox[center,top]{\parbox{12.5cm}{\centering
          \LARGE\color{headcolor}{
            %%
            {\bfseries Simulation of Biological Neuronal Networks}\\[0.3cm]
            %%
            {Introduction:}\\
            %% 
          }
        }}}
    %% 
    \pgfputat{\pgfxy(6.25,4.0)}{\pgfbox[center,top]{\parbox{12.5cm}{\centering
          {\large Robin Pauli, Jyotika Bahuguna,\\ Claudia Bachmann, Abigail Morrison,}\\[0.2cm]
          {\small{} \{r.pauli\textbar c.bachmann\textbar  j.bahuguna \textbar morrison\}@fz-juelich.de{}}
        }}}
        
    %%
    \pgfputat{\pgfxy(6.25,2.0)}{\pgfbox[center,top]{\parbox{12.5cm}{\centering
          \footnotesize
          Inst.~of Neuroscience and Medicine (INM-6)\\
          Computational and Systems Neuroscience\\
          Research Center J\"ulich, Germany
        }}}
    %%
    \pgfputat{\pgfxy(0.5,-0.7)}{\pgfbox[left,bottom]{\parbox{12.5cm}{
          \footnotesize 
some time July        }}}
    %% \showgrid
  }
\end{frame}
%%%%%%%%%%%%%%%%%%%%%%%%%%%%%%%%%%%%%%%%%%%%%%%%%%%%%%%%%%%%%%%%%%%%%%%%%%%

%%%%%%%%%%%%%%%%%%%%%%%%%%%%%%%%%%%%%%%%%%%%%%%%%%%%%%%%%%%%%%%%%%%%%%%%%%%


%%%%%%%%%%%%%%%%%%%%%%%%%%%%%%%%%%%%%%%%%%%%%%%%%%%%%%%%%%%%%%%%%%%%%%%%%%%
%\section{\ttl}
\begin{frame}
  \frametitle{Certificate Requirements}
  \begin{itemize}
  \item Attendance at all lectures (9- $\sim$ 10.30)
  \item Tutor signs off on all daily exercises \\($\sim$ 18:00 is end of day so do it before!)
  \item "Catch" up day on Friday
  \item Course Material at
  \begin{enumerate}
  \item git clone https://github.com/INM-6/BNN\_course\_pub 
  \item git pull
  \item Nest Tutorials: http://www.nest-simulator.org/introduction-to-pynest/
  \end{enumerate}
  \end{itemize}


\end{frame}
%%%%%%%%%%%%%%%%%%%%%%%%%%%%%%%%%%%%%%%%%%%%%%%%%%%%%%%%%%%%%%%%%%%%%%%%%%%

\begin{frame}
\frametitle{Additional Info}
\begin{itemize}
\item After the lecture you may come and go as you please
\item If you are working outside the iLab, you should return by 16:00 at the latest (we need time for checking your exercises)
\item Checklist:
\begin{enumerate}
\item If there's an error: what is the error code telling you?
\item Is it in one of the nest tutorials? (http://www.nest-simulator.org/introduction-to-pynest/)
\item Can I google/stackoverflow it?
\item Does one of the other students have the same problem?
\item Ask the tutors!
\end{enumerate}

\item No food or drink in the iLab
\item Do not leave the iLab empty and unlocked
\end{itemize}

\end{frame}


\end{document}

%%
%% template
%%%%%%%%%%%%%%%%%%%%%%%%%%%%%%%%%%%%%%%%%%%%%%%%%%%%%%%%%%%%%%%%%%%%%%%%%%%
\def\ttl{}
\section{\ttl}
\begin{frame}
  \frametitle{\ttl}
  \pgfslide{
  %%
  %%
  %%\showgrid
  }
\end{frame}
%%%%%%%%%%%%%%%%%%%%%%%%%%%%%%%%%%%%%%%%%%%%%%%%%%%%%%%%%%%%%%%%%%%%%%%%%%%


