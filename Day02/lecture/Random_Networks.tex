%
\documentclass{beamer}
%%\documentclass[handout]{beamer}

\definecolor{headcolor}{rgb}{0,0.3,0.6}
%\renewcommand{\alert}[1]{\textcolor{headcolor}{#1}}
\renewcommand{\alert}[1]{\textcolor{blue}{#1}}
\newcommand{\red}[1]{\textcolor{red}{#1}}
\mode<presentation>
{
  \usetheme{default}
  %% \setbeamercovered{transparent}
  \usefonttheme{professionalfonts}
  \usefonttheme{structurebold}
  \usecolortheme[rgb={0,0.3,0.6}]{structure}
  %%  \usecolortheme[rgb={0,0,0.7}]{structure}
}

\usepackage[english]{babel}
\usepackage[latin1]{inputenc}
\usepackage{times}
\usepackage[T1]{fontenc}
\usepackage{rotate}
\usepackage{color}
\usepackage{verbatim}
\usepackage{xmpmulti}
\usepackage{pgfarrows}
%\usepackage{pgf}

%\input{/home/tom/lib/tex/defs}
%%%%%%%%%%%%%%%%%%%%%%%%%%%%%%%%%%%%

% Delete this, if you do not want the table of contents to pop up at
% the beginning of each subsection:
%\AtBeginSection[]
%{
%  \begin{frame}<beamer>
%    \frametitle{Outline}
%    \tableofcontents[currentsection]
%  \end{frame}
%}
% \AtBeginSubsection[]
% {
%   \begin{frame}<beamer>
%     \frametitle{Outline}
%     \tableofcontents[currentsubsection]
%   \end{frame}
% }

% If you wish to uncover everything in a step-wise fashion, uncomment
% the following command: 
%\beamerdefaultoverlayspecification{<+->}

%%%%%%%%%%%%%%%%%%%%%%%%%%%%%%%%%%%%%%%%%%%%%%%%%%%%%%%%%%%%%%%%%%%%%%%%%%%%%%%

\def\width{12.5} % slide width
\def\height{8.5} % slide height
\newcommand{\showgrid}{%
  \pgfsetlinewidth{0.8pt} 
  \pgfgrid[step={\pgfpoint{1cm}{1cm}}]{\pgforigin}{\pgfxy(\width,\height)}{} 
  \pgfsetlinewidth{0.1pt} 
  \pgfgrid[stepx=0.1cm,stepy=0.1cm]{\pgforigin}{\pgfxy(\width,\height)}
}
\newcommand{\pgfslide}[1]{%
  \hspace*{-1cm}
  \begin{pgfpicture}{0cm}{0cm}{\width cm}{\height cm}
    #1
  \end{pgfpicture}  
}
\def\figpath{figs/}


%%%%%%%%%%%%%%%%%%%%%%%%%%%%%%%%%%%%
\newcommand{\A}{{\text{A}^{-}}}
\newcommand{\Ca}{{\text{Ca}^{2+}}}
\newcommand{\Cl}{{\text{Cl}^{-}}}
\newcommand{\di}{\text{d}}
\newcommand{\diff}[2]{\displaystyle\frac{\text{d}#1}{\text{d}#2}}
\newcommand{\e}{\text{e}}
\newcommand{\Epop}{\mathcal{E}}
\newcommand{\EW}[2][]{\text{E}_{#1}\left[#2\right]}
\newcommand{\Ex}{\text{E}}
\newcommand{\FT}{\mathfrak{F}}
\newcommand{\iFT}{\mathfrak{F}^{-1}}
\newcommand{\Fourier}[2]{\mathfrak{F}\left[#1\right]\left(#2\right)}
\newcommand{\iFourier}[2]{\mathfrak{F}^{-1}\left[#1\right]\left(#2\right)}
\newcommand{\In}{\text{I}}
\newcommand{\inp}{\text{inp}}
\newcommand{\Ipop}{\mathcal{I}}
\newcommand{\K}{{\text{K}^{+}}}
\newcommand{\Leak}{{\text{L}}}
\newcommand{\mm}{\text{mm}}
\newcommand{\mV}{\text{mV}}
\newcommand{\Na}{{\text{Na}^{+}}}
\newcommand{\one}{\mathbb{I}}
\newcommand{\rest}{\text{rest}}
\newcommand{\tauM}{\tau_\text{m}}
\newcommand{\tauR}{\tau_\text{ref}}
\newcommand{\vect}[2]{\begin{pmatrix}#1\\#2\end{pmatrix}}
%%%%%%%%%%%%%%%%%%%%%%%%%%%%%%%%%%%%

%%%%%%%%%%%%%%%%%%%%%%%%%%%%%%%%%%%%%%%%%%%%%%%%%%%%%%%%%%%%%%%%%%%%%%%%%%%%%%%
\begin{document}
%%
%% title page
\begin{frame}
  \pdfbookmark[2]{Title}{TitlePage}
  \pgfslide{
    %% 
    \pgfputat{\pgfxy(6.25,8)}{\pgfbox[center,top]{\parbox{12.5cm}{\centering
          \LARGE\color{headcolor}{
            %%
            {\bfseries Simulation of Biological Neuronal Networks}\\[0.3cm]
            %%
            {Cortical networks:}\\
            {Background and simple models}
            %% 
          }
        }}}
    %% 
    \pgfputat{\pgfxy(6.25,4.0)}{\pgfbox[center,top]{\parbox{12.5cm}{\centering
          {\large Jyotika Bahuguna, Abigail Morrison}\\[0.2cm]
          {\small{}morrison@fz-juelich.de{}}
        }}}
    %%
    \pgfputat{\pgfxy(6.25,2.5)}{\pgfbox[center,top]{\parbox{12.5cm}{\centering
          \footnotesize
          Inst.~of Neuroscience and Medicine (INM-6)\\
          Computational and Systems Neuroscience\\
          Research Center J\"ulich, Germany
        }}}
    %%
    \pgfputat{\pgfxy(0.5,-0.7)}{\pgfbox[left,bottom]{\parbox{12.5cm}{
          \footnotesize 
          February 4th 2015
        }}}
    %% \showgrid
  }
\end{frame}
%%%%%%%%%%%%%%%%%%%%%%%%%%%%%%%%%%%%%%%%%%%%%%%%%%%%%%%%%%%%%%%%%%%%%%%%%%%
\def\ttl{Outline}
\pdfbookmark[2]{Outline}{Outline}
\begin{frame}
  \frametitle{\ttl}
  \tableofcontents
\end{frame}
%%%%%%%%%%%%%%%%%%%%%%%%%%%%%%%%%%%%%%%%%%%%%%%%%%%%%%%%%%%%%%%%%%%%%%%%%%%
\def\ttl{Biological background: Cortex structure}
\section{\ttl}

%%%%%%%%%%%%%%%%%%%%%%%%%%%%%%%%%%%%%%%%%%%%%%%%%%%%%%%%%%%%%%%%%%%%%%%%%%%
%\section{\ttl}
\begin{frame}
  \frametitle{\ttl}
  \pgfslide{
    %% 
    \only<1->{
      \pgfputat{\pgfxy(0,8.5)}{\pgfbox[left,top]{\parbox{12.5cm}{\centering
            \begin{itemize}
            \item \textbf{Macroscopic structure:}
              \begin{itemize}
              %\item relatively undeveloped in lower vertebrates 
              %  (fish, amphibia, reptiles)
              %\item largest part of the mammalian brain 
              \item visible with your eyes 
              \item thousand/millions of cells forming a structure
              \item e.g brain areas, long range connections 
	      \end{itemize}
	      \item \textbf{Microscopic structure:}
                \begin{itemize}
                \item magnifying instruments are needed to recognize structure 
                \item depending on resolution: group of cells, single cell, molecules 
                \item e.g small neuronal network, short range connections 
                \end{itemize}
            \end{itemize}

          }}}
    }
    %% 
    %% \showgrid
  }
\end{frame}
%%%%%%%%%%%%%%
%%%%%%%%%%%%%%%%%%%%%%%%%%%%%%%%%%%%%%%%%%%%%%%%
\def\ttl{Microscopic structure}
%%%%%%%%%%%%%%%%%%%%%%%%%%%%%%%%%%%%%%%%%%%%
\begin{frame}
  \frametitle{\ttl: Cell types}
  \pgfslide{
    %% 
    \only<1->{
      \pgfputat{\pgfxy(0,8.5)}{\pgfbox[left,top]{\parbox{12.5cm}{\centering
            \begin{itemize} 
            \item<1-> cortical tissue is mainly composed of two cell types: 
              \begin{itemize}
              \item {\bfseries neuroglia (glia)}:
                \begin{itemize}
                \item important role in development of the brain
                \item metabolic supportive role
                \item control ionic composition of extracellular space
                \item form myelin sheath around axons of neurons
                \item do not take part in interaction between neurons on a millisecond scale 
                (however, may play a role in slow modulations)
                \end{itemize}
              \item {\bfseries nerve cells (neurons)}:
                \begin{itemize}
                \item carry out information processing and storage in the brain
                \end{itemize}
              \end{itemize}
            \end{itemize}
          }}}
    }
  %%
  %%\showgrid
  }
\end{frame}
\begin{frame}
  \frametitle{\ttl: Neuron types}
  \pgfslide{
  %%
    \only<2>{
      \pgfputat{\pgfxy(6.5,4)}{
        \def\picname{crtcncs_fig1_2_2}
        \pgfdeclareimage[interpolate=true,width=5cm]{\picname}{\figpath\picname}
        \pgfbox[center,center]{\pgfuseimage{\picname}}
      }
      \pgfputat{\pgfxy(6.5,8.2)}{\pgfbox[center,top]{Pyramidal cell}}    
    }
    \only<4>{
      \pgfputat{\pgfxy(6.5,4)}{
        \def\picname{crtcncs_fig1_2_3}
        \pgfdeclareimage[interpolate=true,width=5cm]{\picname}{\figpath\picname}
        \pgfbox[center,center]{\pgfuseimage{\picname}}
      }
      \pgfputat{\pgfxy(6.5,8.2)}{\pgfbox[center,top]{Spiny stellate cell}}    
    }
    \only<6>{
      \pgfputat{\pgfxy(6.5,4)}{
        \def\picname{crtcncs_fig1_2_4}
        \pgfdeclareimage[interpolate=true,width=5cm]{\picname}{\figpath\picname}
        \pgfbox[center,center]{\pgfuseimage{\picname}}
      }
      \pgfputat{\pgfxy(6.5,8.2)}{\pgfbox[center,top]{Smooth stellate cell}}    
    }
    \pgfputat{\pgfxy(0,8.5)}{\pgfbox[left,top]{\parbox{12.5cm}{\centering
          \begin{itemize}
          \item<1,3,5> \alert{pyramidal cell}: 
            \begin{itemize}
            \item dendrites covered with spines
            \item axon leaves cortex into white matter but has numerous branches 
              close to cell body
            \item axons makes \red{excitatory} synapses
            \item cell receives input 
              from inhibitory cells mainly at the soma and from excitatory cells at the basal 
              and apical dendrite
            \end{itemize}
          \item<3,5>\alert{spiny stellate cell}:
            \begin{itemize}
            \item axon branches out within the cortex (rarely leaves the cortex)
            \item axons makes \red{excitatory} synapses
            \item soma receives exclusively inhibitory and dendrites mostly excitatory inputs
            \end{itemize}
          \item<5> \alert{smooth stellate cell} (e.g.~basket cell):
            \begin{itemize}
            \item axon branches out only in the cortex
            \item axons makes \red{inhibitory} synapses (GABA)
            \item both the soma and the dendrites receive a mixture of excitatory 
              and inhibitory inputs
            \end{itemize}
          \end{itemize}        
        }}}
    %% 
  %%\showgrid
  }
\end{frame}
\begin{frame}
  \frametitle{\ttl: Neuron types}
  \pgfslide{
  %%
    \only<1->{
      \pgfputat{\pgfxy(6.4,5)}{
        \def\picname{Binzegger04_8441_fig2}
        \pgfdeclareimage[interpolate=true,width=12.5cm]{\picname}{\figpath\picname}
        \pgfbox[center,center]{\pgfuseimage{\picname}}
      }
      \pgfputat{\pgfxy(12.5,1.5)}{\pgfbox[right,top]{\tiny (Binzegger et al., 2004)}}
    }
  %%\showgrid
  }
\end{frame}
%%%%%%%%%%%%%%%%%%%%%%%%%%%%%%%%%%%%%%%%%%%%%%%%%%%%%%%%%%%%%%%%%%%%%%%%%%%
\begin{frame}
  \frametitle{\ttl: Cortex layers}
  \pgfslide{
  %%
    \only<1->{
      \pgfputat{\pgfxy(3.0,2.5)}{
        \def\picname{layers}
        \pgfdeclareimage[interpolate=true,width=4.2cm]{\picname}{\figpath\picname}
        \pgfbox[center,center]{\pgfuseimage{\picname}}
      }
      \pgfputat{\pgfxy(9.2,3)}{
        \def\picname{crtcncs_fig1_3_4}
        \pgfdeclareimage[interpolate=true,width=6.5cm]{\picname}{\figpath\picname}
        \pgfbox[center,center]{\pgfuseimage{\picname}}
      }
      \pgfputat{\pgfxy(7,0.5)}{\pgfbox[center,top]{\tiny primary motor area}}
      \pgfputat{\pgfxy(11.7,0.5)}{\pgfbox[center,top]{\tiny primary sensory area}}
    }
    %%
    \pgfputat{\pgfxy(0,8.5)}{\pgfbox[left,top]{\parbox{12.5cm}{\centering
          \begin{itemize}
          \item subdivision of neocortex into 6 (or 5) layers
            \begin{itemize}
            \item \alert{layer I}: 
              very few neurons, only dendrites and mesh of horizontal axons
            \item \alert{layer II/III}: 
              small pyramidal cells
            \item \alert{layer IV}: 
              small and medium-size pyramidal cells and stellate cells 
              in layer IVA, 
              almost exclusively stellate cells in IVB
            \item \alert{layer V}: 
              all cell types, very large pyramidal cells dominate
            \item \alert{layer VI}: 
              small and medium-size pyramidal cells
            \end{itemize}
          \end{itemize}
        }}}
  %%\showgrid
  }
\end{frame}
%%%%%%%%%%%%%%%%%%%%%%%%%%%%%%%%%%%%%%%%%%%%%%%%%%%%%%%%%%%%%%%%%%%%%%%%%%%
%%%%%%%%%%%%%%%%%%%%%%%%%%%%%%%%%%%%%%%%%%%%%%%%%%%%%%%%%%%%%%%%%%%%%%%%%%%
\def\ttl{Cortex connectivity: Summary - some numbers}

\begin{frame}
  \frametitle{\ttl}
  \pgfslide{
  %%
    \only<1->{
      \pgfputat{\pgfxy(2.9,4.5)}{
        \def\picname{axons}
        \pgfdeclareimage[interpolate=true,height=7.5cm]{\picname}{\figpath\picname}
        \pgfbox[center,center]{\pgfuseimage{\picname}}
      }
      \pgfputat{\pgfxy(9.3,6.5)}{
        \def\picname{crtcncs_tab1_5_1}
        \pgfdeclareimage[interpolate=true,width=6.5cm]{\picname}{\figpath\picname}
        \pgfbox[center,center]{\pgfuseimage{\picname}}
      }
      \pgfputat{\pgfxy(9.3,2.3)}{
        \def\picname{crtcncs_tab1_5_4}
        \pgfdeclareimage[interpolate=true,width=6.5cm]{\picname}{\figpath\picname}
        \pgfbox[center,center]{\pgfuseimage{\picname}}
      }}
      \only<2->
      {\color{red}
        \pgfsetlinewidth{1pt}
        \pgfellipse[stroke]{\pgfxy(7.0,6.2)}{\pgfxy(1.3,0)}{\pgfxy(0,2)}
        \pgfellipse[stroke]{\pgfxy(11.9,1.95)}{\pgfxy(0.7,0)}{\pgfxy(0,0.19)}
        \pgfellipse[stroke]{\pgfxy(11.75,1.17)}{\pgfxy(0.5,0)}{\pgfxy(0,0.5)}
      }
  %%\showgrid
  }
\end{frame}
%%%%%%%%%%%%%%%%%%%%%%%%%%%%%%%%%%%%%%%%%%%%%%%%%%%%%%%%%%%%%%%%%%%%%%%%%%%
\def\ttl{Some graph-theoretical concepts}
\section{\ttl}
\begin{frame}
  \frametitle{\ttl}
  \pgfslide{
  %%
     \only<1->{
       \pgfputat{\pgfxy(10.8,7)}{
         \def\picname{graph}
         \pgfdeclareimage[interpolate=true,width=4cm]{\picname}{\figpath\picname}
         \pgfbox[center,center]{\pgfuseimage{\picname}}
       }
     }
     \only<3->{
       \pgfputat{\pgfxy(10.8,2)}{
         \def\picname{wgraph}
         \pgfdeclareimage[interpolate=true,width=4cm]{\picname}{\figpath\picname}
         \pgfbox[center,center]{\pgfuseimage{\picname}}
       }
     }
    \only<1->{
      \pgfputat{\pgfxy(0,8.5)}{\pgfbox[left,top]{\parbox{12.5cm}{\centering\small
            \begin{itemize}
            \item<1-> Graph: $G:=(V,E)$\\[0.1cm]
              $V$: set of 'vertices' (neurons)\\
              $E$: set of (ordered) pairs of vertices 
              $\curvearrowright$ 'edges' (synapses)\\[0.2cm]
              example:\\
              $V=\{a,b,c\}$\quad,\quad$E=\{ \{a,b\}  , \{b,a\} , \{b,c\} , \{a,c\} \}$
            \item<2-> more convenient:
            \item<2->[] \alert{adjacency matrix}
              \begin{math}A=
                \begin{pmatrix}
                  0&1&0\\
                  1&0&0\\
                  1&1&0
                \end{pmatrix}
              \end{math}
              \qquad (columns=axons, rows=dendrites)\\[0.2cm]
              with 
              \begin{math}
                A_{ij}=
                \begin{cases}
                  1 & \text {edge (synapse) $j\to{}i$ present}\\
                  0 & \text{else}
                \end{cases}
              \end{math}
            \item<3->[] \alert{weight matrix}
              \begin{math}W=
                \begin{pmatrix}
                  0&J&0\\
                  J&0&0\\
                  J&-J&0
                \end{pmatrix}
              \end{math}\\[0.2cm]
              (for weighted graphs)
            \end{itemize}
          }}}
    }
    %% 
     \only<2->{
       \pgfputat{\pgfxy(5.15,5.3)}{\pgfbox[center,center]{
           \scriptsize\color{red}presynaptic}}
       \pgfputat{\pgfxy(6.2,4.5)}{\pgfbox[center,center]{
           \rotatebox{90}{\scriptsize\color{red}postsynaptic}}}
     }
  %%
  %%\showgrid
  }
\end{frame}
\begin{frame}
  \frametitle{\ttl}
  \pgfslide{
    %% 
    \only<1->{
      \pgfputat{\pgfxy(0,8.5)}{\pgfbox[left,top]{\parbox{12cm}{\centering\small
            \begin{itemize}
            \item<1-> \alert{in-degree}: $K^\text{in}_i=\sum_j A_{ij}$ (number of inputs)
            \item<1->[] \alert{out-degree}: $K^\text{out}_j=\sum_i A_{ij}$ (number of targets)
            \item<2-> note: 
              \begin{itemize}
              \item<2-> $A$ and $W$ generally NOT symmetric
              \item<3-> 'space' not \emph{explicitly} represented (depends on neuron labeling)
              \end{itemize}
            \item<4-> structure of $W$ determines features of network dynamics
            \item<5->[]example: linear firing-rate network model (Wilson-Cowan model)\\[0.2cm]
              \centering\begin{math}\displaystyle
                \tau\frac{dy}{dt}=-y+Wy  
              \end{math}\\[0.2cm]
              with firing rates $y(t)=[y_1(t),\ldots,y_N(t)]^\text{T}$ of neurons $1,\ldots,N$
              \\[0.2cm]
              $\curvearrowright$
              stability of fixed point determined by eigenvalue spectrum of $W-1$
            \end{itemize}
          }}}
    }
    %% 
    %%\showgrid
  }
\end{frame}
%%%%%%%%%%%%%%%%%%%%%%%%%%%%%%%%%%%%%%%%%%%%%%%%%%%%%%%%%%%%%%%%%%%%%%%%%%%
\def\ttl{Random networks}
\section{\ttl}
\begin{frame}
  \frametitle{\ttl}
  \pgfslide{
  %%
    \only<8>{
      \pgfputat{\pgfxy(6.2,4.5)}
      {
        \def\picname{Song05_fig4B.png}
        \pgfdeclareimage[interpolate=true,width=9.5cm]{\picname}{\figpath\picname}
        \pgfbox[center,center]{\only<1->{\pgfuseimage{\picname}}}
      }
      \pgfputat{\pgfxy(6.2,0.3)}{\pgfbox[center,center]{\parbox{\linewidth}{\centering\footnotesize 
          Song et al.~(2005),
          \alert{Highly Nonrandom Features}\\ of Synaptic Connectivity in Local 
          Cortical Circuits
        }}}
    }
    %% 
    \only<1->{
      \pgfputat{\pgfxy(0,8.5)}{\pgfbox[left,top]{\parbox{12cm}{\centering\small
            \begin{itemize}
            \item<1-7,9-> connection matrix $W$ often treated as \emph{random} matrix
              with defined statistical properties, e.g.
              connection probability,
              in- or out-degree distribution,
              weight distribution, \ldots
              (reflects lack of knowledge)
            \item<2-7,9->[] \alert{\large
                There is no such thing as {\bf THE} random-network model!}
            \item<3-7,9->[] Be precise when describing network structure!
            \item<4-7,9-> popular random networks in Computational Neuroscience:
              \begin{itemize}\small
              \item<4-7,9-> Erd\"os-Renyi graphs: 
                \begin{itemize}\small
                \item Each possible connection is 
                randomly and independently drawn with probability $\epsilon$.
                \end{itemize}
              \item<5-7,9-> random networks with fixed in- (or out-) degree:
                \begin{itemize}\small
                \item random distribution of a fixed number of $1$'s in 
                each row (column) of the adjacency matrix
                \end{itemize}
              \item<6-7,9-> small-world networks:
                \begin{itemize}\small
                \item (deterministic) nearest-neighbour connections
                \item random redistribution of individual links with probability $\alpha$
                \end{itemize}
              \item<7,9-> networks with specified high-order connectivity statistics 
                (motifs)
              \item<9-> \ldots and infinitely many more
              \end{itemize}
            \end{itemize}
          }}}
    }
  %%
  %%\showgrid
  }
\end{frame}
%%%%%%%%%%%%%%%%%%%%%%%%%%%%%%%%%%%%%%%%%%%%%%%%%%%%%%%%%%%%%%%%%%%%%%%%%%%
\def\ttl{the balanced-random-network model {\tiny(brunel, 2000)}}
\section{\ttl}
\begin{frame}
  \frametitle{\ttl}
  \pgfslide{
  %%
    \only<1->{
      \pgfputat{\pgfxy(3.3,5.5)}
      {
        \def\picname{brunel}
        \pgfdeclareimage[interpolate=true,width=5cm]{\picname}{\figpath\picname}
        \pgfbox[center,center]{\only<1->{\pgfuseimage{\picname}}}
      }
    }
    \only<1-2>{
      \pgfputat{\pgfxy(9,6.2)}{
        \def\picname{dalemat}
        \pgfdeclareimage[interpolate=true,width=4cm]{\picname}{\figpath\picname}
        \pgfbox[center,center]{\pgfuseimage{\picname}}
      }
      \pgfputat{\pgfxy(9.1,8.1)}{\pgfbox[center,center]{\footnotesize 
          Connection matrix $J=\{J_{ij}\}$}}
      \pgfputat{\pgfxy(9.1,3.8)}{\pgfbox[center,center]{\footnotesize pre}}
      \pgfputat{\pgfxy(6.8,6.2)}{\pgfbox[center,center]{
          \rotatebox{90}{\parbox{2cm}{\centering\footnotesize post}}}}
    }


    \pgfputat{\pgfxy(0,3.7)}{
      \pgfbox[left,top]{
        \parbox{12cm}{
          \footnotesize
          \begin{itemize}
          \item<1-2> $N_\Ex$ excitatory, $N_\In$ inhibitory neurons,
            $N=N_\Ex+N_\In\sim{}10^{4\ldots{}5}$, $N_\Ex$:$N_\In$=4:1
          \item<1-2> random Dale-conform connectivity with fixed in-degrees $K_{\Ex/\In}=N_{\Ex/\In}/10$
          \item<2> LIF dynamics with current synapses\\[0.2cm]
	  \item free parameters: $g$ (ratio of inhibitory to excitatory synaptic weight), $\frac{\nu_{\text{ext}}}{\nu_{\text{thr}}}$ (ratio of external rate to threshold rate)
%            \begin{math}\displaystyle
              %\boxed{
%                \begin{array}{l}
%                  \alert{\tauM{}\dot{v}_i=-v_i(t)+RI_{i}^\text{net}(t)+RI^\text{ext}(t)}
%                  \quad(i\in\{1,\ldots,N\})\\[0.1cm]
%                  \alert{I_{i}^\text{net}(t)=\sum_{j}(h_{ij}*s_j)(t)}
%                  \quad\text{
%                    (with postsynaptic current kernel $h_{ij}(t)=J_{ij}\cdot{}h(t)$)}\\[0.1cm]
%                  \text{if $v_i(t_{i,k})\ge\theta$: spike at time $t_{i,k}$, 
%                    reset $v_i(t_{i,k}^+)=0$ }\\[0.1cm]
%                  \text{spike train: }s_i(t)=\sum_k{}\delta(t-t_{i,k})
%                \end{array}
              %}
%            \end{math}

            %%
          \end{itemize}
        }
      }
    }
    %%
  %%
  %%\showgrid
  }
\end{frame}
%%%%%%%%%%%%%%%%%%%%%%%%%%%%%%%%%%%%%%%%%%%%%%%%%%%%%%%%%%%%%%%%%%%%%%%%%%%
\def\ttl{Activity regimes}
\begin{frame}
  \frametitle{\ttl}
  \pgfslide{
  %%
    \pgfputat{\pgfxy(3,2.45)}
    {
      \def\picname{brunel00_fig2A}
      \pgfdeclareimage[interpolate=true,width=5.5cm]{\picname}{\figpath\picname}
      \rotatebox{0}{\pgfbox[center,center]{\only<2->{\pgfuseimage{\picname}}}}
    }
    %%
    \pgfputat{\pgfxy(9.2,4.5)}
    {
      \def\picname{brunel00_fig8}
      \pgfdeclareimage[interpolate=true,width=6.8cm]{\picname}{\figpath\picname}
      \rotatebox{0}{\pgfbox[center,center]{\only<2>{\pgfuseimage{\picname}}}}
    }
    \pgfputat{\pgfxy(9,2.4)}
    {
      \def\picname{brunel00_fig7}
      \pgfdeclareimage[interpolate=true,width=5.8cm]{\picname}{\figpath\picname}
      \rotatebox{0}{\pgfbox[center,center]{\only<3->{\pgfuseimage{\picname}}}}
    }
    \pgfputat{\pgfxy(7.3,4.7)}{\pgfbox[bottom,left]{\only<3->{\footnotesize\bfseries simulation}}}
    \pgfputat{\pgfxy(1.5,4.7)}{\pgfbox[bottom,left]{\only<2->{\footnotesize\bfseries theory}}}
    \pgfputat{\pgfxy(1,0)}{\pgfbox[bottom,left]{\only<2->{\tiny(Brunel, 2000)}}}
  %%
   \pgfputat{\pgfxy(0,8.5)}
    {
      \pgfbox[left,top]{
        \parbox{6cm}
        {
          \footnotesize
          \begin{itemize}
          \item<1-> analysis of network dynamics in diffusion approximation (small synaptic weights, high firing rates, small correlations)
            {\tiny (see Brunel, 2000)}
          \item<2->[$\curvearrowright$] stationary and oscillatory states\\[0.2cm]
            \tiny
            \begin{tabular}{rl}
              SR:& synchronous regular\\
              AI:& asynchronous irregular\\
              SI:& synchronous irregular (slow and fast)
            \end{tabular}
          \end{itemize}
        }
      }
    }
  %%\showgrid
  }
\end{frame}
%%%%%%%%%%%%%%%%%%%%%%%%%%%%%%%%%%%%%%%%%%%%%%%%%%%%%%%%%%%%%%%%%%%%%%%%%%%
\def\ttl{the balanced-random-network model {\tiny(brunel, 2000)}}
\begin{frame}
  \frametitle{\ttl}
    \only<2-4>{
      %\pgfputat{\pgfxy(9.5,4.2)}
      \pgfputat{\pgfxy(8.5,-1.3)}
      {
        \def\picname{brunel_allstates}
        \pgfdeclareimage[interpolate=true,width=5.5cm]{\picname}{\figpath\picname}
        \pgfbox[center,center]{\only<1->{\pgfuseimage{\picname}}}
      }
      \pgfsetlinewidth{1pt}
      \pgfsetendarrow{\pgfarrowtriangle{2pt}}
      \pgfline{\pgfxy(12.2,1.1)}{\pgfxy(12.2,7.9)}
      \pgfputlabelrotated{.5}{\pgfxy(12.2,1.05)}{\pgfxy(12.2,7.95)}{-2pt}%
      {\pgfbox[center,top]{\footnotesize external drive}}
      \pgfputat{\pgfxy(8.5,2.6)}{\pgfbox[center,bottom]{%
            \footnotesize Activity states
          }} 
    }
    \only<2-4>{
      \pgfsetlinewidth{2pt}
      {
        \color{blue}
        \pgfrect[stroke]{\pgfxy(5.8,-2.65)}{\pgfxy(5.3,2.45)}}
    }
    \only<5->{
      \pgfputat{\pgfxy(8.6,-1.0)}
      {
        \def\picname{chaos}
        \pgfdeclareimage[interpolate=true,width=5.5cm]{\picname}{\figpath\picname}
        \pgfbox[center,center]{\pgfuseimage{\picname}}
      }
      \pgfputat{\pgfxy(9.0,-4.5)}{\pgfbox[center,top]{%
          \parbox{5cm}{ \tiny Simulation of two identical networks
            \mbox{($N\sim{}10000$)}, slight perturbation of initial
            membrane potential of {\color{blue} one} neuron
            \mbox{($\Delta{}V=0.1\,\mV$)} }}} 
      \pgfputat{\pgfxy(8.7,2.4)}{\pgfbox[center,bottom]{%
            \footnotesize Chaos in random networks
          }}
    }
    %%
    \pgfputat{\pgfxy(0,8.5)}
    {
      \pgfbox[left,top]{
        \parbox{7cm}
        {
          \footnotesize

          \begin{itemize}
          \item[]<1->\hspace*{-0.5cm}Simple model of a local cortex volume\\
            \hspace*{-0.5cm}($\sim{}1\text{mm}^3$, $\sim{}10^{4\ldots{}5}$ neurons)
          \end{itemize}
        }
      }
    }
    %%

    %%
    \pgfputat{\pgfxy(-0.5,1.6)}
    {
      \pgfbox[left,top]{
        \parbox{6.5cm}
        {
          \footnotesize
          \begin{itemize}
          \item<2-> Cortical neuron receives much higher proportion
            of excitatory synapses as compared to inhibitory synapses
          \item<3-> In-vivo like activity\\
            - large membrane potential \\fluctuations\\
            - low firing rates\\
            - irregular spiking
          \item<4-> Possible explanation: cortex operates in inhibition dominated regime, i.e
            stronger synaptic strengths for inhibitory neurons \\



          %\item<3-> global oscillatory modes in various frequency bands
          \item<5-> chaotic spiking (high sensitivity)
          %\item[]<1->
          %  \hspace*{-0.5cm}
          %  \tiny (e.g.~van Vreeswijk \& Sompolinsky, 1996 ; Brunel, 2000)
          \end{itemize}
        }
      }
    }





\end{frame}
%%%%%%%%%%%%%%%%%%%%%%%%%%%%%%%%%%%%%%%%%%%%%%%%%%%%%%%%%%%%%%%%%%%%%%%%%%%
\def\ttl{Literature}
\section{\ttl}
\begin{frame}
 \frametitle{\ttl: Cortex anatomy}
  \pgfslide{
  %%
 \pgfputat{\pgfxy(0,8.2)}{\pgfbox[left,top]{\parbox{12.5cm}{\centering
       \begin{itemize}\footnotesize
       \item Abeles,\\ 
         {\it Corticonics: Neural Circuits of the Cerebral Cortex},\\
         Cambridge University Press, 1991
       \item Braitenberg \& Sch{\"u}z,\\
         {\it Cortex: Statistics and Geometry of Neuronal Connectivity},\\
         Springer, 2nd edition, 1998\\[0.5cm]
       \item Thomson \& Bannister (2003),
         Interlaminar Connections in the neocortex,
         Cerebral Cortex 13:5--14
       \item Binzegger et al.~(2004),
         A Quantitative Map of the Circuit of Cat 
         Primary Visual Cortex,
         J Neurosci 39(24):8441--8453
       \item Hellwig (2000), A quantitative analysis of the local connectivity between 
         pyramidal neurons in layers 2/3 of the rat visual cortex,
         Biological Cybernetics 82:111--121
       \item Lund et al.~(2003), Anatomical Substrates for Functional Columns in 
         Macaque Monkey Primary Visual Cortex,
         Cerebral Cortex 13:15--24
       \item Voges et al.~(2010), 
         A modeler's view on the spatial structure of intrinsic horizontal connectivity 
         in the neocortex, Progress in Neurobiology 92(3):277--292
       \item Song et al.~(2005),
         Highly nonrandom features of synaptic connectivity in local cortical circuits,
         PLoS Biology 3(3):0507--0519
       \end{itemize}
     }}}
  %%
  %%\showgrid
  }
\end{frame}
\begin{frame}
 \frametitle{\ttl: Spiking random-network models}
  \pgfslide{
    %%
    \pgfputat{\pgfxy(0,8)}
    {
      \pgfbox[left,top]{
        \parbox{12cm}
        {
          \footnotesize
          \begin{enumerate}
            %%
          \item[]\underline{Random matrices:}
            %%
          \item ML Mehta, {\it Random matrices}, 3rd edition, Elsevier Ltd., 2004
            %%
            \\[0.5cm]
          \item[]\underline{Recurrent-network dynamics:}
            %%
          \item LF Abbott, C van Vreeswijk, Asynchronous states in a network of pulse-coupled oscillators,
            Phys Rev E 48:1483--1490, 1993
            %%
          \item N Brunel, Dynamics of sparsely connected networks of
            excitatory and inhibitory spiking neurons, J Comput
            Neurosci 8:183--208, 2000
            %%
          \item A Renart, N Brunel, XJ Wang, Mean-field theory of
            irregularly spiking neuronal populations and working
            memory in recurrent cortical networks, in
            {\it Computational neuroscience -- A comprehensive approach}, J Feng (ed.), Chapman \& Hall/CRC, London, 2003
            %%
            %%
          \end{enumerate}
        }
      }
    }
    %%
    %%
%    \showgrid
  }
\end{frame}

\end{document}

%%
%% template
%%%%%%%%%%%%%%%%%%%%%%%%%%%%%%%%%%%%%%%%%%%%%%%%%%%%%%%%%%%%%%%%%%%%%%%%%%%
\def\ttl{}
\section{\ttl}
\begin{frame}
  \frametitle{\ttl}
  \pgfslide{
  %%
  %%
  %%\showgrid
  }
\end{frame}
%%%%%%%%%%%%%%%%%%%%%%%%%%%%%%%%%%%%%%%%%%%%%%%%%%%%%%%%%%%%%%%%%%%%%%%%%%%


